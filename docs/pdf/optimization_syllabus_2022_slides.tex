\documentclass{beamer}
%
% Choose how your presentation looks.
%
% For more themes, color themes and font themes, see:
% http://deic.uab.es/~iblanes/beamer_gallery/index_by_theme.html
%
\mode<presentation>
{
	\usetheme{Boadilla}      % or try Darmstadt, Madrid, Warsaw, ...
	\usecolortheme{lily} % or try albatross, beaver, crane, ...
	\usefonttheme{serif}  % or try serif, structurebold, ...
	\setbeamertemplate{caption}[numbered]
} 

\usepackage[english]{babel}
\usepackage{amsmath}
\usepackage{amssymb}
\usepackage{multimedia}
\usepackage{mathpazo}
\usepackage{multirow}
\usepackage{amsthm,sgame}
\usepackage{color} 
\usepackage{hyperref}
\usepackage[UTF8]{ctex}


%\setbeamercolor{block title}{use=alerted text,fg=white,bg=alerted text.fg!75!black}
%\setbeamercolor{block body}{parent=normal text,use=block title alerted,bg=block title alerted.bg!10!bg}
%\setbeamercolor{local structure}{fg=gray}

\setbeamercolor{block title}{use=structure,fg=structure.fg,bg=structure.fg!20!bg}
\setbeamercolor{block body}{parent=normal text,use=block title,bg=block title.bg!50!bg}

\usepackage{tikz,xcolor}
\usetikzlibrary{automata, positioning}
\usetikzlibrary{fit} % for drawing the rectangular
\usetikzlibrary{decorations.pathreplacing}


\definecolor{dark-blue}{rgb}{0,0,0.5}
\hypersetup{colorlinks,linkcolor=dark-blue}
%\usepackage[colorlinks,linkcolor=dark-blue,pagebackref=false]
\addtobeamertemplate{footline}{\hypersetup{allcolors=.}}{}

%backup slides
\newcommand{\backupbegin}{
	\newcounter{framenumberappendix}
	\setcounter{framenumberappendix}{\value{framenumber}}
}
\newcommand{\backupend}{
	\addtocounter{framenumberappendix}{-\value{framenumber}}
	\addtocounter{framenumber}{\value{framenumberappendix}} 
}
	
%% footnote without marker
\newcommand\blfootnote[1]{%
	\begingroup
	\renewcommand\thefootnote{}\footnote{#1}%
	\addtocounter{footnote}{-1}%
	\endgroup
}	
	

\title[Optimization]{\large 动态优化 \\ Dynamic Optimization}
\author[Xiaoxiao Hu]{Xiaoxiao Hu \\ \smallskip \footnotesize Wuhan University}
\date[Feb. 15, 2022]{February 15, 2022}
\begin{document}

\begin{frame}
 \titlepage
\end{frame}
 

% Uncomment these lines for an automatically generated outline.
%\begin{frame}{Outline}
%  \tableofcontents
%\end{frame}


\begin{frame}{Instructor}
\begin{itemize}
	\item Instructor: Xiaoxiao Hu (胡枭骁)
	\item How to reach me?
	\begin{itemize}
		\item In class
		\item Email: \href{mailto:sherryecon@qq.com}{sherryecon@qq.com} 
		\item Office: 经管院 C151 (by appointment)
	\end{itemize}
\end{itemize}
\end{frame}

\begin{frame}{TA}
	\begin{itemize}
		\item TA: Yiqing Zhang (张一清)
		\item She will help answer your questions, including but not limited to questions on assignments.
		\item How to reach her?
	\begin{itemize}
		\item QQ
	\end{itemize}
	\end{itemize}
\end{frame}

\begin{frame}{Course Logistics}
	\begin{itemize}
		\item Class Room: 枫 - 210
		\item Class Hours: Tuesday 14:05 - 16:30 (6 - 8节)
		\item Duration: February 15 - May 31 (Week 1 - 16)		
	\end{itemize}
\end{frame}

\begin{frame}{Course Description}
\begin{itemize}
	\item Undergraduate Course
	\item Foundation knowledge in dynamic optimization 
	\item Topics include Lagrange's Method, Concave Programming, Uncertainty, Dynamic Programming, etc.
	\item Your background?
\end{itemize}
\end{frame}

\begin{frame}{Course Materials}
\begin{columns}
	\begin{column}{0.7\textwidth}
		\begin{itemize}
			\item Textbook: Avinash K. Dixit, Optimization in Economic Theory. Oxford University Press, 1990.		
			\item Course Website: \url{https://xhu.github.io/teaching/optimization2022}
			\item 可以不买教材,教师会在每堂课前公布相应课件。
		\end{itemize}
	\end{column}
	\begin{column}{0.3\textwidth}
		\begin{center}
			\includegraphics[width=\textwidth]{dixit.png}      
		\end{center}
	\end{column}
\end{columns}
\end{frame}

\begin{frame}{Assessments}
	\begin{itemize}
	\item Assignments: 20\%
	\begin{itemize}
		\item There will be 5 assignments in total.
		\item The assignments will mainly be graded based on effort.
	\end{itemize}
	\item Midterm Exam: 30\% 
			\begin{itemize}
				\item Tentative Date: April 19 (Week 10), in-class
			\end{itemize}	
	\item Final Exam: 50\%
			\begin{itemize}
				\item Tentative Date: May 31 (Week 16), in-class
			\end{itemize}
\end{itemize}
\end{frame}

\begin{frame}{Most Important!}
\begin{itemize}
	\item Course Website: \url{https://xhu.github.io/teaching/optimization2022}
	\item QQ group
\end{itemize}


\end{frame}

\begin{frame}
\centering Questions?
\end{frame}
\end{document}
